% !Mode:: "TeX:UTF-8"

% !TEX root = ../main.tex

\chapter{绪论}[Intrudocution]

\section{课题的来源}[Subject Source]

在日常生活中,我们经常会不经意地泄露自己的身份信息,尤其是当我们进行一些与身份验证相关的操作的时候。
想象这样一个场景,当我们走进酒吧去买酒时,通常需要证明自己的年龄已经大于十八岁。
最直接的证明方式就是使用自己的身份证,这种做法尽管可以完成验证,但同时也存在着许多安全隐患。
身份证中包含了完整的个人信息,这种直接出示的方式无疑将自己的身份信息暴露出去,从而造成个人隐私泄露的问题。

重新考虑一下这个场景,不难发现,用户只需要提供“年龄”这一个属性信息。
如果用户能够在不提供完整的身份凭证的情况下,证明自己的年龄确实满足大于十八岁的条件,那么他就能最大限度地保护自己其它的身份信息不被泄露。
一般情况下,当用户申请获取某项服务时,服务提供商需要用户证明其拥有获取此项服务的权限。
一种保护隐私的方法是,用户只需要提供一种记录着获取这项服务所需的必要属性的凭证,服务提供商可以凭借此凭证完成对用户权限的认证。
像这种能够在不暴露额外个人隐私信息的情况下完成对属性所有权的证明的凭证,我们就称之为基于属性的凭证(Attributed-Based Credentials,ABCs)。

\section{课题的研究背景和意义}[Significance]

随着信息技术的不断发展,互联网已经成为了我们日常生活中不可或缺的一部分。
由于网络的普及,各类互联网产品层出不穷。
现有的互联网服务已经覆盖了衣食住行各个方面,我们足不出户便能享受互联网带来的便利。
在这种信息产业高度发达的环境下,也存在着不少的安全隐患。
尤其是以数字化形式存储的个人信息,相比于传统的纸质信息,更容易被互联网服务提供商收集、存档甚至出售给第三方组织,从而导致个人隐私信息泄露事件频频发生。
比如近几年频繁发生的多起信息诈骗案件,究其根源就在于用户个人隐私信息遭到泄露。
目前国内的网络黑灰产业链已达千亿规模$\footnote{http://finance.people.com.cn/n1/2018/0823/c1004-30245368.html}$,这些网络黑灰产行业利用一些漏洞非法获取用户的信息,并通过收集用户的上网习惯,勾勒出其在现实生活中的真实状态,然后实施有针对性的诈骗。
除此之外,有些商业公司也会利用用户的隐私信息来为自己牟利。
2018年,Facebook公司就被指出通过收集用户的个人资料数据并将这些数据出售给剑桥分析公司$\footnote{https://wallstreetcn.com/articles/3256863}$。
剑桥分析公司使用这些数据在美国总统大选期间来推送具有倾向性的广告,间接地影响美国大选。
这类事件的频繁发生,更加凸显保护隐私的重要性。

在大多数场景下,互联网服务提供商没有必要收集用户的隐私信息。
那么我们是什么情况下提交了自己的个人隐私数据呢?
通常是向服务提供商验证自己的身份以获取某项服务的时候。
因此我们需要关注的是如何在不泄露隐私的情况下,完成对拥有此项服务使用权限的证明。
这也是隐私保护领域面临的新挑战。

基于属性的凭证正是为了解决这类问题而提出的。
当我们在证明自己拥有某项服务的使用权限时,可以根据服务提供商要求的基本属性,提供相应的基于属性的凭证。
这类凭证与传统的身份凭证不同,它只记录了用户的一些属性信息,服务提供商通过验证此类凭证来确定用户是否具备相应权限。
仅仅通过基于属性的凭证,服务提供商无法将其与用户的真实身份关联起来。
这种方式能够在最大程度上保护用户的隐私,从根源上防止用户个人隐私数据的泄露。
从服务提供商的角度来说,使用这类的凭证系统能很好地取得用户的信任,从而吸引更多的用户使用他们提供的服务。
这种以匿名的形式进行交互的方式在现实生活场景中有很广泛的应用。
目前的绝大多数服务提供商都不可避免地收集用户的个人信息,而且随着大数据、物联网及数据挖掘等技术的发展,用户也正面临着越来越多的安全隐患。
基于属性的匿名凭证一旦得到广泛应用,可以有效地保护用户的个人隐私。

但是要构建一个这样的系统是一件棘手的事,需要结合许多密码学的理论和工具。
传统的构造方案主要关注于协议的设计,距离实际应用场景还有一定的距离。
现有的基于属性的凭证系统已经能很好地实现匿名等性质,但在效率方面还有待提高。
随着一些新的密码学原语的提出,基于属性的凭证系统的构造方案也有了新的进展。
如何设计一个高效且与具有实际应用价值的基于属性的凭证系统,是目前热门的研究内容。

\section{国内外研究进展及成果}[Progress]

基于属性的凭证是由匿名凭证(Anonymous Credentials,ACs)发展而来,第一个真正意义上可行的匿名凭证系统IdeMix是由Camenisch与Herreweghen于2002年提出来的\cite{camenisch2002design}。
经过十多年的发展,许多不同的方案相继被提出。
但由于研究的视角不同,国内外研究工作的侧重点也不同。

国外在隐私保护方面的研究发展较早,处于领先地位。
目前国外的研究工作主要集中在设计具有实际意义的基于属性的凭证系统平台,如ABC4Trust\cite{sabouri2012attribute}(Attribute-Based Credentials for Trust)、IRMA\cite{vullers2013efficient}(I Reveal My Attributes)。
与传统的IdeMix系统不同,这些方案不再停留在协议的设计,更多的是提供了一套完整的开发框架,开发者可以通过使用这些平台来开发自己的应用。
另外,一些新颖的匿名凭证如去中心化的匿名凭证\cite{garman2014decentralized},可委托的基于属性的凭证\cite{blomer2018delegatable}等具有特殊应用场景的方案也相继被提出。
与此同时,还出现了将匿名凭证系统应用到智慧城市\cite{de2017assessment},智能交通\cite{neven2017privacy}和物联网\cite{viejo2019secure}等特定应用场景中的构想。

相较于国外,国内对这类的研究相对较少,比较相关的有一些像环签名、聚合签名等具有匿名性质的签名方案\cite{张严2012匿名凭证方案研究进展,胡江红2017可证明安全的基于证书聚合签名方案},但没有一般化的匿名凭证的构造方案。
另外,国内也有关于直接匿名证明(Direct Anonymous Attestation)的方案的构建\cite{张严2012匿名凭证方案研究进展},并有基于直接匿名证明方案在电子现金系统等商务系统方面具体应用\cite{柳欣2016基于DAA-A的改进可授权电子现金系统,柳欣2016基于DAA的轻量级多商家多重息票系统}。

\section{主要研究内容}[Main Content]

关于本课题的主要研究内容,我们可以从以下三个部分分别进行阐述。

第一部分,我们将主要介绍实现匿名凭证的密码学基础。
要构造一个匿名凭证系统,许多密码学原语需要被用到。
这些密码学原语包括数字签名(Digital Signature)、零知识证明(Zero-Knowledge Proofs)、承诺机制(Commitment Schemes)及密码哈希函数(Cryptographic Hash Function)等。
尤其是对于数字签名,还需要进一步介绍具备特殊性质的签名方案,如盲签名(Blind Signature)、群签名(Group Signature)、环签名(Ring Signature)以及基于属性的签名(Attribute-Based Signature)等。
在这一部分,我们还将着重介绍实现这些数字签名的椭圆曲线公钥密码机制,包括双线性映射(Bilinear Map)及其相关的性质等内容。

对于第二部分,首先我们将调研匿名凭证的相关工作,着重关注近些年来有关基于属性的凭证的理论原型,并进一步分析这些原型的优势与劣势,以及在现实生活场景中的应用。
同时我们将关注近些年比较成熟的基于属性的凭证系统,解析现有方案的构建框架。

第三部分的重点是对应用场景的研究。
具体地说,我们将考虑车辆自组网的环境,并对该环境中存在的安全与隐私保护问题进行充分调研。
通过分析现有的一些方案,找到尚待解决的问题,再结合之前的研究工作,尝试把基于属性的凭证系统中的优势用到车辆自组网的环境中,提出我们的解决方案。

此后,我们还要对提出的方案进行充分的理论分析。
一方面要在分析其优势的基础上,对满足的性质给出具体的证明。
在另一方面,通过具体的实验测试,分析其效率,并与其它相近方案做比较,最后对已完成的工作进行总结。