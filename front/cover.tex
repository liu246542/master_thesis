% !Mode:: "TeX:UTF-8"
% !TEX root = ../main.tex

\hitsetup{
  %******************************
  % 注意:
  %   1. 配置里面不要出现空行
  %   2. 不需要的配置信息可以删除
  %******************************
  %
  %=====
  % 秘级
  %=====
  statesecrets={公开},
  natclassifiedindex={TP393.1},
  intclassifiedindex={654},
  %
  %=========
  % 中文信息
  %=========
  ctitleone={基于属性的凭证},%本科生封面使用
  ctitletwo={及其应用},%本科生封面使用
  ctitlecover={基于属性的凭证及其应用},%放在封面中使用,自由断行
  ctitle={基于属性的凭证及其应用},%放在原创性声明中使用
  % csubtitle={一条副标题}, %一般情况没有,可以注释掉
  cxueke={工学},
  csubject={计算机科学与技术},
  caffil={南方科技大学},
  cauthor={柳枫},
  csupervisor={王琦助理教授},
  % cassosupervisor={某某某教授}, % 副指导老师
  % ccosupervisor={某某某教授}, % 联合指导老师
  % 日期自动使用当前时间,若需指定按如下方式修改:
  cdate={2019年6月},
  cstudentid={11749258},
  cstudenttype={硕士}, %非全日制教育申请学位者
  %(同等学力人员)、(工程硕士)、(工商管理硕士)、
  %(高级管理人员工商管理硕士)、(公共管理硕士)、(中职教师)、(高校教师)等
  %
  %
  %=========
  % 英文信息
  %=========
  etitle={Attribute-based credentials: Theory and Applications},
  % esubtitle={This is the sub title},
  exueke={Engineering},
  esubject={Computer Science and Technology},
  eaffil={\emultiline[t]{Southern University of Science and Technology}},
  eauthor={Feng Liu},
  esupervisor={Prof. Qi Wang},
  % eassosupervisor={XXX},
  % 日期自动生成,若需指定按如下方式修改:
  edate={June, 2019},
  estudenttype={Master of Art},
  %
  % 关键词用“英文逗号”分割
  ckeywords={匿名凭证, 基于属性的凭证, 环签名, 车辆自组网},
  ekeywords={anonymous credentials, attribute-based credentials, ring signature, VANETs},
}

\begin{cabstract}

在信息产业高度发达的今天,人们在享受数字化服务带来便利的同时,服务提供商也在不断收集用户的隐私信息。
如在申请某项服务时,用户通常需要提供身份信息来证明自己拥有获取此项服务的权限,服务商也就同时获取到该身份信息。
要保护用户的隐私,我们可以使用一种具有匿名性质的凭证来代替传统的身份凭证。
基于属性的凭证就属于这类具有匿名性质的凭证,它可以保证用户在完成验证的同时不泄露自己的隐私。
像这样的凭证系统通常需要具备匿名性和可验证性,我们可以通过密码学来实现这两个基本性质。

在这篇论文中,我们首先介绍了一些重要的密码学原语,然后阐述了如何利用这些密码学原语构造基于属性的凭证系统,并简要介绍了现实中存在的系统实例。
作为基于属性的凭证系统的一个重要应用,我们针对车辆自组网中的安全与隐私保护问题,并结合现有的基于属性的凭证系统的构造及特点,设计了一个高效的保护隐私的解决方案。
此方案的核心是基于环签名的机制,在满足车辆自组网环境中安全需求的情况下,同时有效地保护车辆的隐私。
与现有的其它基于环签名的方案不同,此方案利用了基于身份的密码学和对称密码学的优势,合理地将路边的通信设施融入到整个系统中,解决了其它基于环签名的方案中存在的列表成员的可验证性问题。
通过与其它方案对比,我们总结了该提出方案的优势,并对所实现的性质进行了详细说明。

\end{cabstract}

\begin{eabstract}

With the rapid development of information technology, while people are enjoying the convenience enabled by information technology, the service providers are collecting users' private information.
For example, when a user requests service, the identity information is usually required as a proof of his/her subscription while the service providers can learn the user's identity.
For the purpose of preserving privacy in certain authentication scenarios, a kind of anonymous credentials are proposed.
Attribute-based credentials (ABCs) are one of these anonymous credentials, which provide a powerful tool to preserve privacy in applications, of which the two important properties: authenticity and anonymity are achieved using cryptography.

In this thesis, we first introduce some important cryptographic primitives, and show how to use these primitives to construct an ABCs system. 
Then we focus on the vehicular ad-hoc networks(VANETs) and propose our scheme,which is efficient and privacy-preserving.
Our proposed scheme is based on ring signature scheme, and is very different from other existing schemes by combining ID-based cryptography and symmetric cryptography, and is efficient to fulfill the verification of ring members.
Furthermore, we compare our proposed scheme with other existing schemes, and summarize the properties of our scheme.

\end{eabstract}