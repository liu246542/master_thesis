% !Mode:: "TeX:UTF-8" 
% !TEX root = ../main.tex

\begin{conclusions}

在研究基于属性的凭证系统的过程中,我们了解到许多解决隐私保护问题的方法。
这些方法与现代密码学紧密相连。
在现代密码学中有许多有用的工具,如数字签名、零知识证明、承诺机制等等,通过将这些工具结合起来,我们可以解决实际生活中的具体问题。

实际生活中有很多亟待解决的安全与隐私问题,基于属性的凭证就是用于解决用户在进行身份认证的场景下的隐私保护问题。
我们详细介绍了基于属性的凭证的由来以及它的发展过程,通过调研大量的文献资料,我们对匿名凭证需要满足的性质进行了总结,并对匿名凭证系统中各个角色的交互过程进行了归纳。
通过对现有的匿名凭证系统如IdeMix、U-Prove和ABC4Trust等的探讨,我们概括了匿名凭证系统的发展规律,并逐步理解构造基于属性凭证背后所用到的密码学思想。

沿着这条思路,我们开始寻找相关的应用场景。
通过对车辆自组网环境的充分调研,我们发现了其中存在的隐私与安全问题。
目前这些问题还没有很好的解决方案,因此我们决定利用基于属性的凭证的思想来对这个问题进行深入的研究。

以往的解决方案中,车辆需要频繁地更换自己的假名以达到匿名的目的。
假名的生成方式要么由TRC完成,要么由车辆自己生成。
这种频繁更换的方式还不能真正意义上地阻止对身份的追踪,因此一些基于特殊性质的签名方案(如群签名、环签名)的解决方法开始被提出。

我们的方案也是基于环签名的,这样最大的好处是增加了假名的利用率,减少了更换假名的频率。
但与其它基于环签名方案最大的不同点在于,其它基于环签名的方案中没有考虑到成员的有效性问题,由于环签名本身的高自由度,这样生成的签名很有可能是失效的。

因此,我们通过基于身份的密码体制,将RSUs很好地融入到系统中来,充当分配成员列表的角色。
最终我们提出了一个结合了多个密码体制的方案。此方案主要有以下几个特点:
\begin{itemize}
  \item[1.] 缩减了更换假名的频率,从而避免因为频繁更换假名带来的计算和存储瓶颈;
  \item[2.] 提出了一个高效的密钥分配方案,只需要一次交互便可以完成通信双方的密钥协商,并利用对称加密机制完成数据的通信;
  \item[3.] 利用执法机关与TRC合作还原出签名者的真实身份,这样的做法有利于维护签名者的隐私,更符合现实生活中的实际应用场景。
\end{itemize}

由于我们使用的是比较轻量级的加密与签名方案,因此在效率方面要比其他基于环签名的方案高。
我们对方案中所实现的性质进行了充分的证明,即在配备有安全模块HSM的假设前提下,我们提出的方案具备可验证性、匿名性和可追踪性等基本性质,并且我们还证明在随机预言机模型下具备不可伪造性。

除了这些优势之外,我们方案还有许多需要完善的地方。
比如在安全性上要依赖HSM,我们应该考虑在没有HSM的场景中如何确保能满足这些性质。
而在随机预言机模型下的安全性证明只能表明系统在某种意义上是安全的,还可以考虑在没有随机预言机的情况下完成安全性证明。
在进行撤销方面,我们没有进行太多的讨论,或许可以尝试使用基于累加器的方法来提高撤销的效率。
在最后的效率分析中,我们缺少更有说服力的仿真测试,这些都是需要我们将来进一步完成的。
\end{conclusions}
